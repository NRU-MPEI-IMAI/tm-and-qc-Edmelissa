\documentclass[a4paper, 12pt]{article}
\usepackage[16pt]{extsizes}
\usepackage{mathtext}
\usepackage[T1,T2A]{fontenc}
\usepackage[english,russian]{babel}
\usepackage[left=12.7mm, top=12.7mm, right=12.7mm, bottom=12.7mm, nohead, footskip=5mm]{geometry} % настройки полей документа

\usepackage{amsfonts}

%для вставки рисунков
\usepackage{graphicx}
\graphicspath{ {./images/} }

\usepackage[fleqn]{amsmath}

\title{Домашняя работа №3}
\author{А-13а-19 Самсонова Мария}

\begin{document}

\maketitle

\section{Реализуйте машины Тьюринга, которые позволяют выполнять следующие операции:}
\begin{enumerate}
    \item Сложение двух унарных чисел 
    \newline
    Граф решения МТ
    \newline
    \includegraphics[scale=0.5]{1_1}
    \newline
    Проверка работоспобособности алгоритма
    \newline
    \begin{tabular}{|*{3}{c|}}
       \hline
       № & входная строка & выходная строка \\
       \hline
       1 & \includegraphics[scale=0.4]{1_1(1.1)} & \includegraphics[scale=0.4]{1_1(1.2)} \\ 
       \hline
       2 & \includegraphics[scale=0.4]{1_1(2.1)} & \includegraphics[scale=0.4]{1_1(2.2)} \\ 
       \hline
       3 & \includegraphics[scale=0.4]{1_1(3.1)} & \includegraphics[scale=0.4]{1_1(3.2)} \\
       \hline
    \end{tabular}
    
    \item Умножение унарных чисел 
    \newline
    Граф решения МТ
    \newline
    \includegraphics[scale=0.5]{1_2}
    \newline
    Проверка работоспобособности алгоритма
    \newline
    \begin{tabular}{|*{3}{c|}}
       \hline
       № & входная строка & выходная строка \\
       \hline
       1 & \includegraphics[scale=0.4]{1_2(1.1)} & \includegraphics[scale=0.4]{1_2(1.2)} \\ 
       \hline
       2 & \includegraphics[scale=0.4]{1_2(2.1)} & \includegraphics[scale=0.4]{1_2(2.2)} \\ 
       \hline
       3 & \includegraphics[scale=0.4]{1_2(3.1)} & \includegraphics[scale=0.4]{1_2(3.2)} \\
       \hline
    \end{tabular}
    
  \item Принадлежность к языку $L = \{ 0^n1^n2^n \}, n \ge 0$
   \newline
    Граф решения МТ
    \newline
    \includegraphics[scale=0.5]{1_3}
    \newline
    Проверка работоспобособности алгоритма
    \newline
    \begin{tabular}{|*{3}{c|}}
       \hline
       № & входная строка & выходная строка \\
       \hline
       1 & \includegraphics[scale=0.4]{1_3(1.1)} & \includegraphics[scale=0.4]{1_3(1.2)} \\ 
       \hline
       2 & \includegraphics[scale=0.4]{1_3(2.1)} & \includegraphics[scale=0.4]{1_3(2.2)} \\ 
       \hline
       3 & \includegraphics[scale=0.4]{1_3(3.1)} & \includegraphics[scale=0.4]{1_3(3.2)} \\
       \hline
       4 & \includegraphics[scale=0.4]{1_3(4.1)} & \includegraphics[scale=0.4]{1_3(4.2)} \\
       \hline
       5 & \includegraphics[scale=0.4]{1_3(5.1)} & \includegraphics[scale=0.4]{1_3(5.2)} \\
       \hline
    \end{tabular}
  
  \item Проверка соблюдения правильности скобок в строке (минимум 3 вида скобок) 
   \newline
    Граф решения МТ
    \newline
    \includegraphics[scale=0.5]{1_4}
    \newline
    Проверка работоспобособности алгоритма
    \newline
    \begin{tabular}{|*{3}{c|}}
       \hline
       № & входная строка & выходная строка \\
       \hline
       1 & \includegraphics[scale=0.4]{1_4(1.1)} & \includegraphics[scale=0.4]{1_4(1.2)} \\ 
       \hline
       2 & \includegraphics[scale=0.4]{1_4(2.1)} & \includegraphics[scale=0.4]{1_4(2.2)} \\ 
       \hline
       3 & \includegraphics[scale=0.4]{1_4(3.1)} & \includegraphics[scale=0.4]{1_4(3.2)} \\
       \hline
       4 & \includegraphics[scale=0.4]{1_4(4.1)} & \includegraphics[scale=0.4]{1_4(4.2)} \\
       \hline
    \end{tabular}
  
  \item Поиск минимального по длине слова в строке (слова состоят из символов 1 и 0 и разделены пробелом) 
   \newline
    Граф решения МТ
    \newline
    \includegraphics[scale=0.5]{1_5}
    \newline
    Проверка работоспобособности алгоритма
    \newline
    \begin{tabular}{|*{3}{c|}}
       \hline
       № & входная строка & выходная строка \\
       \hline
       1 & \includegraphics[scale=0.4]{1_5(1.1)} & \includegraphics[scale=0.4]{1_5(1.2)} \\ 
       \hline
       2 & \includegraphics[scale=0.4]{1_5(2.1)} & \includegraphics[scale=0.4]{1_5(2.2)} \\ 
       \hline
       3 & \includegraphics[scale=0.4]{1_5(3.1)} & \includegraphics[scale=0.4]{1_5(3.2)} \\
       \hline
       4 & \includegraphics[scale=0.4]{1_5(4.1)} & \includegraphics[scale=0.4]{1_5(4.2)} \\
       \hline
    \end{tabular}
\end{enumerate}

\end{document}
